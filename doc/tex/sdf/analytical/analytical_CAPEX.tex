\clearpage

\section{CAPEX}\label{analytical_CAPEX}
\begin{tcolorbox}	
\begin{tabular}{p{2.75cm} p{0.2cm} p{10.5cm}} 	
\textbf{Student Name}  &:& Tiago Esteves    (October 03, 2017 - )\\
\textbf{Goal}          &:& Implement of the analytical model to obtain the best possible CAPEX of a given network.
\end{tabular}
\end{tcolorbox}
\vspace{11pt}

The cost of a telecommunications network can be divided into CAPEX and OPEX.
CAPEX is the amount of money needed to set up and install a particular network.
OPEX is the amount of money needed to run this network as well as its maintenance and operation over time.
In this section we will only focus on CAPEX, that is, the costs of installing a particular network.
As we know the telecommunications networks are made up of links and nodes, so it is possible to define the CAPEX as being the sum of the cost of links and cost of nodes.
This can be said that the CAPEX cost in monetary units, $C_C$ is given by the equation \ref{analytical_Capex}

\begin{equation}
C_C = C_L + C_N
\label{analytical_Capex}
\end{equation}

\vspace{11pt}
\noindent
where $C_L$ is the Link cost and $C_N$ is the Node cost.\\

For this calculation first let's focus on the cost of the links. Where to calculate the cost of the Links, $C_L$, we will use the equation \ref{analytical_linkCosts}

\begin{equation}
C_L = \left(2 L \gamma_0^{OLT}\right) + \left(2 L \gamma_1^{OLT} \tau <w>\right) + \left(2 N^R c^R\right)
\label{analytical_linkCosts}
\end{equation}

\vspace{11pt}
\noindent
where
\begin{itemize}
\item{$\gamma_0^{OLT}$	$\rightarrow$	OLT cost in euros}
\item{$L$				$\rightarrow$	Number of bidirectional links}
\item{$\gamma_1^{OLT}$	$\rightarrow$	Transponder cost in euros}
\item{$<w>$             $\rightarrow$   Average number of optical channels}
\item{$\tau$		    $\rightarrow$	Traffic per port}
\item{$N^R$				$\rightarrow$	Total number of optical amplifiers}
\item{$c^R$				$\rightarrow$	Unidirectional Optical amplifiers cost in euros}
\end{itemize}

\vspace{11pt}
Looking at the equation \ref{analytical_linkCosts} we can see that we already have practically all the values of the variables used. Assuming that $\tau$ is 100 Gbits/s is thus only missing the number of optical amplifiers and the average number of optical channels.\\

Through the equation \ref{amplifiers} we can calculated the number of optical amplifiers, $N^R$, as

\begin{equation}
N^R = \sum\limits_{l=1}^L\left(\left\lceil\frac{len_l}{span}\right\rceil-1\right)
\label{amplifiers}
\end{equation}

\vspace{11pt}
\noindent
where $len_l$ is the length of link $l$ and $span$ is the distance between amplifiers (assuming 100 km).\\
Through the equation \ref{optical_channels} we can calculated the average number of optical channels, $<w>$, as

\begin{equation}
<w> = \left( \frac{\lceil D \times <h> \rceil}{L} \right) \left( 1 + <k>\right)
\label{optical_channels}
\end{equation}

\vspace{11pt}
\noindent
where $D$ is the number of unidirectional demands, $L$ is the number of unidirectional Links and $<k>$ is the survivability coefficient.\\
The number os unidirectional demands can be calculated as

\begin{equation}
D = \left(\frac{1}{2}\right) \left( 1 + \xi \right) \left(\frac{T_1}{\tau}\right)
\label{demands}
\end{equation}

\vspace{11pt}
\noindent
where $\xi$ is the grooming coefficient, $T_1$ is the total unidirectional traffic and $\tau$ is the line bit rate.\\

The next step is to take into account the cost of the nodes, but for this we must first know how a node is constituted. The nodes have an electrical part and an optical part so we can conclude that the cost of the nodes, $C_N$, is given by the sum of these two parts thus obtaining the equation \ref{analytical_Capex_Node}

\begin{equation}
C_N = C_{EXC} + C_{OXC}
\label{analytical_Capex_Node}
\end{equation}

\vspace{11pt}
To know the electric cost, $C_{exc}$, of the nodes that is given by equation \ref{analytical_electricalCost}

\begin{equation}
C_{exc} = N \times \left( \gamma_{e0} + \left( \gamma_{e1} \tau <P_{exc}> \right) \right) + \gamma_{e1} P_{TRIB}
\label{analytical_electricalCost}
\end{equation}

\vspace{11pt}
\noindent
where:
\begin{itemize}
\item{$N$			$\rightarrow$	Number of nodes}
\item{$\gamma_{e0}$	$\rightarrow$	EXC cost in euros}
\item{$\gamma_{e1}$	$\rightarrow$	EXC port cost in euros}
\item{$\tau$		$\rightarrow$	Traffic per port}
\item{$<P_{exc}>$   $\rightarrow$   Average number of ports of the electrical switch}
\item{$P_{TRIB}$    $\rightarrow$   Total number of tributary ports}
\end{itemize}

\vspace{11pt}
In relation to the optical part, $C_{oxc}$, to know the optical cost of the nodes that is given by equation \ref{analytical_opticalCost}

\begin{equation}
C_{oxc} = N \times \left( \gamma_{o0} + \left( \gamma_{o1} <P_{oxc}> \right) \right)
\label{analytical_opticalCost}
\end{equation}

\vspace{11pt}
\noindent
where:
\begin{itemize}
\item{$N$			$\rightarrow$	Number of nodes}
\item{$\gamma_{o0}$	$\rightarrow$	OXC cost in euros}
\item{$\gamma_{o1}$	$\rightarrow$	OXC port cost in euros}
\item{$<P_{oxc}>$   $\rightarrow$   Average number of ports of the optical switch}
\end{itemize}

\vspace{11pt}
We have to take into account that the calculated value for the variables $<P_{exc}>$ and $<P_{oxc}>$ will depend on the mode of transport used (opaque, transparent or translucent) and the variable $P_{TRIB}$ will depend on the scenario but later on it will be explained how these values are calculated for each specific transport mode.
Finally, for this we will also have to take into account the cost of the equipment used that can be consulted in table \ref{table_cost_analytical}.\\

\begin{table}[h!]
\centering
\begin{tabular}{|| c | c | c ||}
 \hline
 Equipment & Symbol & Cost \\
 \hline\hline
 OLT without transponders & $\gamma_0^{OLT}$ & 15 000 \euro \\
 Transponder & $\gamma_1^{OLT}$ & 5 000 \euro/Gb \\
 Unidirectional Optical Amplifier & $c^R$ & 4 000 \euro \\
 EXC & $\gamma_{e0}$ & 10 000 \euro \\
 OXC & $\gamma_{o0}$ & 20 000 \euro \\
 EXC Line Ports & $\gamma_{e1}$ & 100 000 \euro /port\\
 EXC Tributary Ports & $\gamma_{e2}$ & 20 \euro /port\\
 OXC Port & $\gamma_{o1}$ & 2 500 \euro /port \\
 \hline
\end{tabular}
\caption{Table with costs}
\label{table_cost_analytical}
\end{table}

