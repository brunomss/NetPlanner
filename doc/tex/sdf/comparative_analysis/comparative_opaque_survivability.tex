\clearpage

\section{Opaque without Survivability}\label{comparative_Opaque_Survivability}
\begin{tcolorbox}	
\begin{tabular}{p{2.75cm} p{0.2cm} p{10.5cm}} 	
\textbf{Student Name}  &:& Tiago Esteves    (October 03, 2017 - )\\
\textbf{Goal}          &:& Comparative analysis of the results of the models used for the opaque transport mode without survivability.
\end{tabular}
\end{tcolorbox}
\vspace{11pt}


In this section we will compare the CAPEX values obtained for the three scenarios in the three types of dimensioning. For a better analysis of the results will be created the table \ref{table_comparative_opaque_sur_ref_1} (scenario 1), the table \ref{table_comparative_opaque_sur_ref_2} (scenario 2) and the table \ref{table_comparative_opaque_sur_ref_3} (scenario 3) with the different values obtained.\\

\textbf{Low traffic scenario:}

\begin{table}[h!]
\centering
\begin{tabular}{| c | c | c | c |}
 \hline
   & Analytical & ILP & Heuristic \\
 \hline\hline
 Link Cost & 16 372 000 \euro & 11 372 000 \euro & 12 004 000 \euro \\
 Node Cost & 5 125 500 \euro & 3 260 000 \euro & 4 360 000 \euro \\
 CAPEX & \textbf{21 497 500 \euro} & \textbf{14 632 000 \euro} & \textbf{16 364 000 \euro} \\
 \hline
\end{tabular}
\caption{Table with different value of CAPEX }
\label{table_comparative_opaque_sur_ref_1}
\end{table}

\vspace{11pt}
Looking at the previous table we can make some comparisons between the several different models of dimensioning and finally draw some conclusions.

\begin{itemize}
  \item We can conclude that in this case the dimensioning using ILP is the best (lowest cost).
  \item In comparison with the analytical model we can see that there is a difference considered with a 32\% error, this value is mainly due to the number of optical channels because in the analytical model more optical channels are calculated and used than those required in the ILP model.
  \item In comparison with the heuristic model we can see that there is a small difference with a 11\% error, much smaller than the previous one.
\end{itemize}

\vspace{11pt}
\textbf{Medium traffic scenario:}

\begin{table}[h!]
\centering
\begin{tabular}{| c | c | c | c |}
 \hline
   & Analytical & ILP & Heuristic \\
 \hline\hline
 Link Cost & 72 372 000 \euro & 35 372 000 \euro & 36 004 000 \euro \\
 Node Cost & 21 337 200 \euro & 11 260 000 \euro & 12 260 000 \euro \\
 CAPEX & \textbf{93 709 200 \euro} & \textbf{46 632 000 \euro} & \textbf{48 264 000 \euro} \\
 \hline
\end{tabular}
\caption{Table with different value of CAPEX }
\label{table_comparative_opaque_sur_ref_2}
\end{table}

\vspace{11pt}
Looking at the previous table we can make some comparisons between the several different models of dimensioning and finally draw some conclusions.

\begin{itemize}
  \item We can conclude that in this case the dimensioning using ILP is the best (lowest cost).
  \item In comparison with the analytical model we can see that there is a difference considered with a 50\% error.
  \item In comparison with the heuristic model we can see that there is a smaller difference with a 3\% error.
\end{itemize}


\vspace{11pt}
\textbf{High traffic scenario:}\\

\begin{table}[h!]
\centering
\begin{tabular}{| c | c | c | c |}
 \hline
   & Analytical & ILP & Heuristic \\
 \hline\hline
 Link Cost & 160 372 000 \euro & 75 372 000 \euro & 76 504 000 \euro \\
 Node Cost & 50 719 500 \euro & 25 060 000 \euro & 25 960 000 \euro \\
 CAPEX & \textbf{211 091 500 \euro} & \textbf{100 432 000 \euro} & \textbf{102 464 000 \euro} \\
 \hline
\end{tabular}
\caption{Table with different value of CAPEX }
\label{table_comparative_opaque_sur_ref_3}
\end{table}

\vspace{11pt}
Looking at the previous table we can make some comparisons between the several different models of dimensioning and finally draw some conclusions.

\begin{itemize}
  \item We can conclude that in this case the dimensioning using ILP is the best (lowest cost).
  \item In comparison with the analytical model we can see that there is a difference considered with a 66\% error.
  \item In comparison with the heuristic model we can see that there is a smaller difference with a 2\% error.
\end{itemize}

