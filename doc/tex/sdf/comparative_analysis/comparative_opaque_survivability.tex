\clearpage

\section{Opaque without Survivability}\label{comparative_Opaque_Survivability}
\begin{tcolorbox}	
\begin{tabular}{p{2.75cm} p{0.2cm} p{10.5cm}} 	
\textbf{Student Name}  &:& Tiago Esteves    (October 03, 2017 - )\\
\textbf{Goal}          &:& Comparative analysis of the results of the models used for the opaque transport mode without survivability.
\end{tabular}
\end{tcolorbox}
\vspace{11pt}


In this section we will compare the CAPEX values obtained for the three scenarios in the three types of dimensioning. For a better analysis of the results will be created the table \ref{table_comparative_opaque_sur_ref_1} (scenario 1), the table \ref{table_comparative_opaque_sur_ref_2} (scenario 2) and the table \ref{table_comparative_opaque_sur_ref_3} (scenario 3) with the different values obtained.\\

\textbf{Low traffic scenario:}

\begin{table}[h!]
\centering
\begin{tabular}{| c | c | c | c |}
 \hline
   & Analytical & ILP & Heuristic \\
 \hline\hline
 Link Cost & 16 380 000 \euro & 9 404 000 \euro & 13 520 000 \euro \\
 Node Cost & 5 125 500 \euro & 1 862 590 \euro & 2 662 590 \euro \\
 CAPEX & \textbf{21 505 500 \euro} & \textbf{11 266 590 \euro} & \textbf{16 182 590 \euro} \\
 \hline
\end{tabular}
\caption{Table with different value of CAPEX }
\label{table_comparative_opaque_sur_ref_1}
\end{table}

\vspace{11pt}
Looking at the previous table we can make some comparisons between the several different models of dimensioning and finally draw some conclusions.

\begin{itemize}
  \item We can conclude that in this case the dimensioning using ILP is the best (lowest cost).
  \item In comparison with the analytical model we can see that there is a difference considered with a 32\% error, this value is mainly due to the number of optical channels because in the analytical model more optical channels are calculated and used than those required in the ILP model.
  \item In comparison with the heuristic model we can see that there is a small difference with a 11\% error, much smaller than the previous one.
\end{itemize}

\vspace{11pt}
\textbf{Medium traffic scenario:}

\begin{table}[h!]
\centering
\begin{tabular}{| c | c | c | c |}
 \hline
   & Analytical & ILP & Heuristic \\
 \hline\hline
 Link Cost & 154 380 000 \euro & 75 520 000 \euro & 94 520 000 \euro \\
 Node Cost & 50 719 500 \euro & 15 085 900 \euro & 18 862 590 \euro \\
 CAPEX & \textbf{205 099 500 \euro} & \textbf{90 605 900 \euro} & \textbf{113 382 590 \euro} \\
 \hline
\end{tabular}
\caption{Table with different value of CAPEX }
\label{table_comparative_opaque_sur_ref_2}
\end{table}

\vspace{11pt}
Looking at the previous table we can make some comparisons between the several different models of dimensioning and finally draw some conclusions.

\begin{itemize}
  \item We can conclude that in this case the dimensioning using ILP is the best (lowest cost).
  \item In comparison with the analytical model we can see that there is a difference considered with a 50\% error.
  \item In comparison with the heuristic model we can see that there is a smaller difference with a 3\% error.
\end{itemize}


\vspace{11pt}
\textbf{High traffic scenario:}\\

\begin{table}[h!]
\centering
\begin{tabular}{| c | c | c | c |}
 \hline
   & Analytical & ILP & Heuristic \\
 \hline\hline
 Link Cost & 307 380 000 \euro & 148 520 000 \euro & 186 020 000 \euro \\
 Node Cost & 101 380 500 \euro & 29 711 800 \euro & 37 162 590 \euro \\
 CAPEX & \textbf{408 760 500 \euro} & \textbf{178 231 800 \euro} & \textbf{223 182 590 \euro} \\
 \hline
\end{tabular}
\caption{Table with different value of CAPEX }
\label{table_comparative_opaque_sur_ref_3}
\end{table}

\vspace{11pt}
Looking at the previous table we can make some comparisons between the several different models of dimensioning and finally draw some conclusions.

\begin{itemize}
  \item We can conclude that in this case the dimensioning using ILP is the best (lowest cost).
  \item In comparison with the analytical model we can see that there is a difference considered with a 66\% error.
  \item In comparison with the heuristic model we can see that there is a smaller difference with a 2\% error.
\end{itemize}

