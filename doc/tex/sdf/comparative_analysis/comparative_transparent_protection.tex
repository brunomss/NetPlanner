\clearpage

\section{Transparent with 1+1 Protection}\label{comparative_Transp_Protection}
\begin{tcolorbox}	
\begin{tabular}{p{2.75cm} p{0.2cm} p{10.5cm}} 	
\textbf{Student Name}  &:& Tiago Esteves    (October 03, 2017 - )\\
\textbf{Goal}          &:& Comparative analysis of the results of the models used for the transparent transport mode with 1 plus 1 protection.
\end{tabular}
\end{tcolorbox}
\vspace{11pt}

In this section we will compare the CAPEX values obtained for the three scenarios in the three types of dimensioning. For a better analysis of the results, the table \ref{table_comparative_transp_protec} was created with all the scenarios used and their values obtained.

\begin{table}[h!]
\centering
\begin{tabular}{| c | c | c | c | c |}
 \hline
 \multicolumn{2}{| c |}{ } & Analytical & ILP & Heuristic \\
 \hline\hline
 \multirow{3}{*}{\textbf{Low Traffic}} & Link Cost & 24 000 800 \euro & 68 520 000 \euro & 68 520 000 \euro \\
  & Node Cost & 1 448 333 \euro & 3 947 590 \euro & 4 007 590 \euro \\
  & CAPEX & \textbf{25 449 133 \euro} & \textbf{72 467 590 \euro} & \textbf{72 527 590 \euro} \\
  \hline
 \hline
 \multirow{3}{*}{\textbf{Medium Traffic}} & Link Cost & 226 198 400 \euro & 226 520 000 \euro & 226 520 000 \euro \\
  & Node Cost & 12 863 336 \euro & 13 020 900 \euro & 15 890 900 \euro \\
  & CAPEX & \textbf{239 061 736 \euro} & \textbf{239 540 900 \euro*} & \textbf{242 410 900 \euro} \\
 \hline
 \hline
 \multirow{3}{*}{\textbf{High Traffic}} & Link Cost & 450 572 800 \euro & 424 520 000 \euro & 424 520 000 \euro \\
  & Node Cost & 25 546 673 \euro & 24 286 800 \euro & 29 821 800 \euro \\
  & CAPEX & \textbf{476 119 473 \euro} & \textbf{448 806 800 \euro*} & \textbf{454 341 800 \euro} \\
 \hline
\end{tabular}
\caption{Transparent with 1+1 protection: Table with different value of CAPEX for all scenarios.}
\label{table_comparative_transp_protec}
\end{table}

Comparing the ILP model with the analytical model for this transport mode with 1 + 1 protection there is a very high margin of error (approximately 64\%) for the low scenario. This error happens again for the same reason as above. In this ILP model the coefficient of grooming varies and in this case this value is once again much higher than the analytic one.
For the other two scenarios, as previously mentioned, due to its complexity the model was only executed during two weeks. After these two weeks is presented the best result found so far, which may be the optimal cost or not.
Still in relation to the analytical mode, for the remaining scenarios, it is possible to conclude that it has a much lower margin of error (below 10\%).
Equating to the heuristic model it is possible to observe that the result obtained by the ILP model is always better (smaller) than the value obtained through the heuristic.
For the medium and high scenarios, although it is not possible to guarantee that the indicated value is optimal, it is possible to affirm that it is quite close since, as previously mentioned, it maintains a margin lower than 10\% (compared to the analytic) and obtained a lower value in relation to heuristic.
