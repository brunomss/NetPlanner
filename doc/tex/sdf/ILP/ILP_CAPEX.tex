\clearpage

\section{CAPEX}
\begin{tcolorbox}	
\begin{tabular}{p{2.75cm} p{0.2cm} p{10.5cm}} 	
\textbf{Student Name}  &:& Tiago Esteves    (October 03, 2017 - )\\
\textbf{Goal}          &:& Implement of the ILP model to obtain the best possible CAPEX of a given network.
\end{tabular}
\end{tcolorbox}
\vspace{11pt}

The cost of a telecommunications network can be divided into CAPEX and OPEX.
CAPEX is the amount of money needed to set up and install a particular network.
OPEX is the amount of money needed to run this network as well as its maintenance and operation over time.
In this section we will only focus on CAPEX, that is, the costs of installing a particular network.
As we know the telecommunications networks are made up of links and nodes, so it is possible to define the CAPEX as being the sum of the cost of links and cost of nodes.
This can be said that the value of CAPEX is given by the sum of equation \ref{Capex_Link} with the equation \ref{Capex_Node}.\\

Link Cost is:\\

\begin{equation}
C_L = \sum_{i=1}^N \sum_{j=i+1}^N L_{ij} \left(2 \gamma_0^{OLT} + 2 \gamma_1^{OLT} \tau W_{ij} + N^R_{ij} c^R \right)
\label{Capex_Link}
\end{equation}


\vspace{10pt}
\begin{itemize}
\item{$C_L$				$\rightarrow$	Links cost}
\item{$i$               $\rightarrow$   Index for start node of a physical link}
\item{$j$               $\rightarrow$   Index for end node of a physical link}
\item{$N$				$\rightarrow$	Total number of nodes}
\item{$L_{ij}$			$\rightarrow$	Binary variable indicating if link between the nodes $i$ and $j$ is used}
\item{$\gamma_0^{OLT}$	$\rightarrow$	OLT cost in euros}
\item{$\gamma_1^{OLT}$	$\rightarrow$	Transponder cost in euros}
\item{$\tau$		    $\rightarrow$	Line bit-rate}
\item{$W_{ij}$          $\rightarrow$   Number of optical channels in link $i$ $j$}
\item{$N^R_{ij}$    	$\rightarrow$	Number of optical amplifiers in link $i$ $j$}
\item{$c^R$				$\rightarrow$	Optical amplifiers cost in euros}
\end{itemize}


The cost of nodes is given by the sum of electrical cost and the optical cost \ref{Capex_Node}\\

\begin{equation}
C_N = C_{EXC} + C_{OXC}
\label{Capex_Node}
\end{equation}

where:

\begin{equation}
C_{EXC} = \sum_{n=1}^{N} N_{exc,n} \left( \gamma_{e0} + \sum_{c=-1}^4 \gamma_{e1,c} P_{exc,c,n} \right)
\label{Capex_Node_EXC}
\end{equation}

\begin{itemize}
\item{$C_{EXC}$			$\rightarrow$	Electrical cost}
\item{$N$				$\rightarrow$	Total number of nodes}
\item{$N_{exc,n}$		$\rightarrow$	Binary variable indicating if node $n$ is used}
\item{$\gamma_{e0}$ 	$\rightarrow$	EXC cost in euros}
\item{$c$               $\rightarrow$   Index for bit rate (-1; 0; 1; 2; 3; 4)}
\subitem{$-1$           $\rightarrow$   100 Gbits/s line bit-rate (long-reach port)}
\subitem{$0$            $\rightarrow$   1.25 Gbits/s tributary bit-rate (short-reach port)}
\subitem{$1$            $\rightarrow$   2.5 Gbits/s tributary bit-rate (short-reach port)}
\subitem{$2$            $\rightarrow$   10 Gbits/s tributary bit-rate (short-reach port)}
\subitem{$3$            $\rightarrow$   40 Gbits/s tributary bit-rate (short-reach port)}
\subitem{$4$            $\rightarrow$   100 Gbits/s tributary bit-rate (short-reach port)}
\item{$\gamma_{e1,c}$	$\rightarrow$	EXC port cost in euros in bit rate $c$}
\item{$P_{exc,c,n}$	    $\rightarrow$	Number of ports of the electrical switch}
\end{itemize}

and:

\begin{equation}
C_{OXC} = \sum_{n=1}^{N} N_{oxc,n} ( \gamma_{o0} + \gamma_{o1} P_{oxc,n} )
\label{Capex_Node_OXC}
\end{equation}

\begin{itemize}
\item{$C_{OXC}$			$\rightarrow$	Optical cost}
\item{$N$				$\rightarrow$	Total number of nodes}
\item{$N_{oxc,n}$		$\rightarrow$	Binary variable indicating if node $n$ is used}
\item{$\gamma_{o0}$ 	$\rightarrow$	OXC cost in euros}
\item{$\gamma_{o1}$ 	$\rightarrow$	OXC port cost in euros }
\item{$P_{oxc,n}$	    $\rightarrow$	Number of ports of the optical switch}
\end{itemize}


\vspace{10pt}
We have to take into account that the calculated value for the variable $P_{exc,c,n}$ and $P_{oxc,n}$ will depend on the mode of transport used (opaque, transparent or translucent) but later on it will be explained how these values are calculated for each specific transport mode.\\
To obtain the best possible value, it will be necessary to minimize the cost of the two equations mentioned above so that we can obtain the objective function.


