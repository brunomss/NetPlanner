\clearpage

\subsection{Opaque without Survivability}\label{ILP_Opaque_Survivability}
\begin{tcolorbox}	
\begin{tabular}{p{2.75cm} p{0.2cm} p{10.5cm}} 	
\textbf{Student Name}  &:& Tiago Esteves    (October 03, 2017 - )\\
\textbf{Goal}          &:& Implement the ILP model for the opaque transport mode without survivability.
\end{tabular}
\end{tcolorbox}

\subsubsection{Model description}

First, for a better understanding of the functions and variables used in the ILP, a table \ref{description_opaque} will be created with all indexes, inputs and variables and with their respective description.\\

\begin{table}[h!]
\centering
\begin{tabular}{ |p{1cm}||p{13cm}|}
 \hline
 \multicolumn{2}{|c|}{Description of notation used in the objective function} \\
 \hline
 \hline
 $i$ & index for start node of a physical link \\
 $j$ & index for end node of a physical link \\
 $o$ & index for node that is origin of a demand \\
 $d$ & index for node that is destination of a demand \\
 $c$ & index for bit rate of the client signal \\
 $($ i,j $)$ & physical link between the nodes $i$ and $j$ \\
 $($ o,d $)$ & demand between the nodes $o$ and $d$ \\
 $C$ & set of the client signal \\
 $f_{ij}^{od}$ & binary variable indicating if link between the nodes $i$ and $j$ is used in the path between nodes $o$ and $d$ \\
 $L_{ij}$ & binary variable indicating if link between the nodes $i$ and $j$ is used \\
 $W_{ij}$ & number of optical channels between the nodes $i$ and $j$\\
 $B_c $ & client signals granularities $($1.25, 2.5, 10, 40, 100$)$ \\
 $D_{odc}$ & client demands with bit rate $c$ between nodes $o$ and $d$ \\
 $G_{ij}$ & network topology in form of adjacency matrix \\
 \hline
\end{tabular}
\caption{Table with description of variables}
\label{description_opaque}
\end{table}

Before carrying out the description of the objective function we must take into account the following particularity of this mode of transport:
\begin{itemize}
  \item $N_{OXC,n}$ = 0, \quad $\forall$ n
  \item $N_{EXC,n}$ = 1, \quad $\forall$ n that process traffic
\end{itemize}


\vspace{11pt}
The objective function of following the ILP is a minimization of the CAPEX through the equation \ref{Capex} where in this case for the cost of nodes we only have in consideration the electric cost \ref{Capex_Node_EXC} because of the particularity previously mentioned.
In this case the value of $P_{exc,c,n}$ is obtained by equation \ref{EXC_pexc1_opaque} for long-reach and by the equation \ref{EXC_pexc2_opaque} for short-reach.\\

\newpage
As previously mentioned, equation \ref{EXC_pexc1_opaque} refers to the number of long-reach ports, that is, the number of line ports of node n is calculated.

\begin{equation}
P_{exc,-1,n} = \sum_{j=1}^{N} w_{nj}
\label{EXC_pexc1_opaque}
\end{equation}

\begin{itemize}
\item{$P_{exc,-1,n}$	$\rightarrow$	Number of long-reach ports of the electrical switch, i.e. number of line ports}
\item{$w_{nj}$			$\rightarrow$	Number of optical channels between node $n$ and node $j$}
\end{itemize}

\vspace{11pt}
As previously mentioned, equation \ref{EXC_pexc2_opaque} refers to the number of sort-reach ports, that is, the number of tributary ports with bit-rate c in node n is calculated.

\begin{equation}
P_{exc,c,n} = \sum_{d=1}^{N} D_{nd,c}
\label{EXC_pexc2_opaque}
\end{equation}

\begin{itemize}
\item{$P_{exc,c,n}$	$\rightarrow$	Number of sort-reach ports of the electrical switch}
\item{$D_{nj,c}$	$\rightarrow$	client demands between nodes $n$ and $d$ with bit rate $c$}
\end{itemize}

\vspace{11pt}
In this case there is the following particularity:

\begin{itemize}
  \item When $n$=$j$ the value of client demands is always zero, i.e, $D_{nn,c}=0$
\end{itemize}


\vspace{17pt}
Objective Function :

\begin{equation}
minimize \qquad \qquad C_C
\label{ILPOpaque_Surv}
\end{equation}

$subject$ $to$
\begin{equation}
\sum_{j\textbackslash \{o\}} f_{ij}^{od} = 1  \qquad \qquad \qquad \qquad \qquad \qquad \qquad \qquad \qquad \qquad
\forall(o,d) : o < d, \forall i: i = o
\label{ILPOpaque1_Surv}
\end{equation}

\begin{equation}
\sum_{j\textbackslash \{o\}} f_{ij}^{od} = \sum_{j\textbackslash \{d\}} f_{ji}^{od}   \qquad \qquad \qquad \qquad \qquad \qquad \qquad \qquad
\forall(o,d) : o < d, \forall i: i \neq o,d
\label{ILPOpaque2_Surv}
\end{equation}

\begin{equation}
\sum_{j\textbackslash \{d\}} f_{ji}^{od} = 1  \qquad \qquad \qquad \qquad \qquad \qquad \qquad \qquad \qquad \qquad
\forall(o,d) : o < d, \forall i: i = d
\label{ILPOpaque3_Surv}
\end{equation}

\begin{equation}
\sum_{(o,d):o<d} \left(f_{ij}^{od} + f_{ji}^{od}\right) + \sum_{c\in C} (B\left(c\right) D_{odc}\leq100 W_{ij} G_{ij} \qquad \qquad \qquad \qquad
\forall(i,j) : i < j
\label{ILPOpaque4_Surv}
\end{equation}

\begin{equation}
W_{ij} \leq 80 L_{ij} \qquad  \qquad \qquad \qquad \qquad \qquad \qquad \qquad \qquad \qquad \qquad \qquad \qquad \forall(i,j) : i < j
\label{ILPOpaque5_Surv}
\end{equation}

\begin{equation}
f_{ij}^{od} , f_{ji}^{od} \in \{0,1\}   \qquad \qquad \qquad \qquad \qquad \qquad \qquad \qquad \qquad
\forall(i,j) : i < j, \forall(o,d) : o < d
\label{ILPOpaque6_Surv}
\end{equation}

\begin{equation}
W_{ij} \in \mathbb{N}  \qquad \qquad \qquad \qquad \qquad \qquad \qquad \qquad \qquad \qquad \qquad \qquad \qquad
\forall(i,j) : i < j
\label{ILPOpaque7_Surv}
\end{equation}

\vspace{10pt}

The objective function, to be minimized, is the expression \ref{ILPOpaque_Surv}. The flow conservation constraints are \ref{ILPOpaque1_Surv}, \ref{ILPOpaque2_Surv} and \ref{ILPOpaque3_Surv}. First constraint ensures that, for all demand pairs (o,d), it routes one flow of traffic for all bidirectional links (i,j) when $j$ is not equal to the origin of the demand. Equation \ref{ILPOpaque3_Surv} is based on the same idea of \ref{ILPOpaque1_Surv}, however applied in reverse direction. Assuming bidirectional traffic, so the number of flows in both directions of the link is the same \ref{ILPOpaque2_Surv}. The inequality \ref{ILPOpaque4_Surv} is considered grooming constraint, so it means the total client traffic flows can not be greater than the capacity of optical channels on all links. Another important constraint \ref{ILPOpaque5_Surv} is the capacity of the optical channels which must be less or equal to 100 Gb/s or 80 ODU0. The number of flows per demand can be zero if there are no traffic demands or one if considering traffic \ref{ILPOpaque6_Surv}. The last constraint \ref{ILPOpaque7_Surv} is just needed to ensure the number optical of channels is a positive integer values greater than zero.\\


\subsubsection{Result description}
