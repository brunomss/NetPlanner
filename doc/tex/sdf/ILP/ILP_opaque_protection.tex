\clearpage

\subsection{Opaque with 1+1 Protection}\label{ILP_Opaque_Protection}
\begin{tcolorbox}	
\begin{tabular}{p{2.75cm} p{0.2cm} p{10.5cm}} 	
\textbf{Student Name}  &:& Tiago Esteves    (October 03, 2017 - )\\
\textbf{Goal}          &:& Implement the ILP model for the opaque transport mode with 1 plus 1 protection.
\end{tabular}
\end{tcolorbox}
\vspace{11pt}

Here, in this case, we must take into account table \ref{description_opaque}, previously mentioned, in order to better understand the objective function.\\

The objective function of following the ILP is a CAPEX minimization using the sum of two variables multiplied by the price previously mentioned: total number of flows crossing the link (i, j) for all demand pairs (o, d) and total number of optical channels in each link (i, j).\\

\begin{equation}
minimize \quad \sum_{(i,j)} \sum_{(o,d)} f_{ij}^{od} + 4000 \times \sum_{(i,j)} L_{ij} + 2 \times 5000 \times \sum_{(i,j)} W_{ij}
\label{ILPOpaque}
\end{equation}

$subject$ $to$
\begin{equation}
\sum_{j\textbackslash \{o\}} f_{ij}^{od} = 2  \qquad \qquad \qquad \qquad \qquad \qquad \qquad \qquad \qquad \qquad
\forall(o,d) : o < d, \forall i: i = o
\label{ILPOpaque1}
\end{equation}

\begin{equation}
\sum_{j\textbackslash \{o\}} f_{ij}^{od} = \sum_{j\textbackslash \{d\}} f_{ji}^{od}   \qquad \qquad \qquad \qquad \qquad \qquad \qquad \qquad
\forall(o,d) : o < d, \forall i: i \neq o,d
\label{ILPOpaque2}
\end{equation}

\begin{equation}
\sum_{j\textbackslash \{d\}} f_{ji}^{od} = 2  \qquad \qquad \qquad \qquad \qquad \qquad \qquad \qquad \qquad \qquad
\forall(o,d) : o < d, \forall i: i = d
\label{ILPOpaque3}
\end{equation}

\begin{equation}
\sum_{(o,d):o<d} \left(f_{ij}^{od} + f_{ji}^{od}\right) + \sum_{c\in C} (B\left(c\right) D_{odc}\leq100 W_{ij} G_{ij} \qquad \qquad \qquad \qquad
\forall(i,j) : i < j
\label{ILPOpaque4}
\end{equation}

\begin{equation}
W_{ij} \leq 80 \qquad  \qquad \qquad \qquad \qquad \qquad \qquad \qquad \qquad \qquad \qquad \qquad \qquad \forall(i,j) : i < j
\label{ILPOpaque5}
\end{equation}

\begin{equation}
\sum_{(o,d)} \left(f_{ij}^{op} + f_{ji}^{op}\right)<= 80 \times L_{ij} \qquad \qquad \qquad \qquad \qquad \qquad \qquad \qquad \qquad \qquad
\forall (i,j)
\label{ILPOpaqueX}
\end{equation}

\begin{equation}
L_{ij} , f_{ij}^{od} , f_{ji}^{od} \in \{0,1\}   \qquad \qquad \qquad \qquad \qquad \qquad \qquad \qquad
\forall(i,j) : i < j, \forall(o,d) : o < d
\label{ILPOpaque6}
\end{equation}

\begin{equation}
W_{ij} \in \mathbb{N}  \qquad \qquad \qquad \qquad \qquad \qquad \qquad \qquad \qquad \qquad \qquad \qquad \qquad
\forall(i,j) : i < j\label{ILPOpaque7}
\end{equation}

\vspace{10pt}

The objective function, to be minimized, is the expression \ref{ILPOpaque}. The flow conservation constraints are \ref{ILPOpaque1}, \ref{ILPOpaque2} and \ref{ILPOpaque3}. First constraint ensures that, for all demand pairs (o,d), it routes two flows of traffic for all bidirectional links (i,j) when $j$ is not equal to the origin of the demand. Equation \ref{ILPOpaque3} is based on the same idea of \ref{ILPOpaque1}, however applied in reverse direction. Assuming bidirectional traffic, so the number of flows in both directions of the link is the same \ref{ILPOpaque2}. The inequality \ref{ILPOpaque4} is considered grooming constraint, so it means the total client traffic flows can not be greater than the capacity of optical channels on all links. Another important constraint \ref{ILPOpaque5} is the capacity of the optical channels which must be less or equal to 100 Gb/s or 80 ODU0. The number of flows per demand can be zero if there are no traffic demands or one if considering working or protection traffic \ref{ILPOpaque6}. The last constraint \ref{ILPOpaque7} is just needed to ensure the number optical of channels is a positive integer values greater than zero.

