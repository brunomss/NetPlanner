\clearpage

\section{Installing LPSOLVE for using in MatLab}
\begin{tcolorbox}	
\begin{tabular}{p{2.75cm} p{0.2cm} p{10.5cm}} 	
\textbf{Student Name}  &:& Tiago Esteves        (November 28, 2017 - December 05, 2017)\\
\textbf{Goal}          &:& Help other to install lpsolve for using in MatLab.
\end{tabular}
\end{tcolorbox}


In this section will describe how to install lpsolve and how it can be used through matlab. For this it is necessary to follow the following steps:
\begin{enumerate}
  \item Install lpsolve in your computer
  \item Install lpsolve matlab extensions
  \item Install the library
\end{enumerate}

\textbf{Step 1:}\\
The first thing to do is to install lpsolve using the execute file "lp-solve-5.5.2.5-IDE-Setup" that can be found in GitHub through this link \url{https://github.com/netxpto/NetPlanner/tree/Develop/software/lpsolve}. The installation is quite simple and contains few steps for its execution. \\
In case there is any doubt or question you can always use the lpsolve reference guide in link: \url{http://lpsolve.sourceforge.net/5.5/} \\

\textbf{Step 2:}\\
In this step it is necessary to go to GitHub again and download with the name "lp-solve-5.5.2.0-MATLAB-exe-win64" in link \url{https://github.com/netxpto/NetPlanner/tree/Develop/software/lpsolve} and extract all the files. Then you need to put the \textbf{mxlpsolve.mexw64} and \textbf{mxlpsolve.m} files in the same folder as the .m files. \\
In case there is any doubt or question you can always use this guide for help: \url{there is any doubt or question you can always use the lpsolve reference guide in link:} \\

\textbf{Setp 3:}\\
Finally, once again in GitHub through the link \url{https://github.com/netxpto/NetPlanner/tree/Develop/software/lpsolve} we can find the last folder to get the necessary library, thus downloading the folder "lp-solve-5.5.2.0-dev-win64" and then include in the Windows PATH environment. \\

