\clearpage

\subsection{Translucent without Survivability}\label{ILP_Transluc_Survivability}
\begin{tcolorbox}	
\begin{tabular}{p{2.75cm} p{0.2cm} p{10.5cm}} 	
\textbf{Student Name}  &:& Tiago Esteves    (October 03, 2017 - )\\
\textbf{Goal}          &:& Implement the ILP model for the translucent transport mode without survivability.
\end{tabular}
\end{tcolorbox}
\vspace{11pt}

First, for a better understanding of the functions and variables used in the ILP, a table \ref{description_transluc} will be created with all the variables and their description. \\

\begin{table}[h!]
\centering
\begin{tabular}{ |p{1cm}||p{13cm}|}
 \hline
 \multicolumn{2}{|c|}{Description of notation used in the objective function} \\
 \hline
 \hline
 $i$ & index for start node of a physical link \\
 $j$ & index for end node of a physical link \\
 $o$ & index for node that is origin of a demand \\
 $d$ & index for node that is destination of a demand \\
 $($ i,j $)$ & physical link between the nodes $i$ and $j$ \\
 $($ o,d $)$ & demand between the nodes $o$ and $d$ \\
 $c$ & Client traffic Type $($ 1 to 5 $)$ \\
 $L_{ij}$ & binary variable indicating if link between the nodes $i$ and $j$ is used \\
 $Ls_{ij}^{od}$ & Number of ODU-o low speed signals from node $o$ to node $d$ employing lightpath ($i$,$j$) \\
 $f_{ij}^{od}$ & Number of 100 Gbit/s optical channels (number of flows) between the link $i$ and $j$ for all demand pairs between $o$ and $d$ \\
 $\lambda_{od}$ & Number of lightpath channels between the nodes $o$ and $d$ \\
 $B$ & Client signals granularities $($1.25, 2.5, 10, 40, 100$)$ \\
 $D_{od}$ & Client traffic demands between the nodes $o$ and $d$ \\
 $G$ & Network topology in form of adjacency matrix \\
 \hline
\end{tabular}
\caption{Table with description of variables}
\label{description_transluc}
\end{table}

The objective function of following the ILP is a minimization of the sum of two equations: the cost of the links \ref{Capex_Link} and cost of the nodes \ref{Capex_Node} where in this case we have in consideration the electric cost and the optical cost.
In this case the value of $P_{exc,c,n}$ is obtained by equation \ref{EXC_pexc_translucent} and the value of $P_{oxc,n}$ is obtained by equation \ref{OXC_poxc_translucent}.

\begin{equation}
P_{exc,c,n} = 2 \tau W
\label{EXC_pexc_translucent}
\end{equation}

\begin{itemize}
\item{$P_{exc,c,n}$	    $\rightarrow$	Number of ports of the electrical switch}
\item{$\tau$			$\rightarrow$	Traffic per port assuming 100Gbits/s}
\item{$W$				$\rightarrow$	Total number of optical channels}
\end{itemize}

\begin{equation}
P_{oxc,n} = P_{ADD} + P_{LINE}
\label{OXC_poxc_translucent}
\end{equation}
\begin{itemize}
\item{$P_{oxc,n}$	    $\rightarrow$	Number of ports of the optical switch}
\item{$P_{ADD}$			$\rightarrow$	Number of adding ports}
\item{$P_{LINE}$		$\rightarrow$	Number of line ports}
\end{itemize}

\vspace{17pt}
The objective function, to be minimized, is the expression \ref{ILPOpaque_CAPEX}.\\

$subject$ $to$
\begin{equation}
\sum_{j\textbackslash \{o\}} Ls_{ij}^{od} = D_{odc}  \qquad \qquad \qquad \qquad \qquad \qquad \qquad \qquad \qquad
\forall(o,d) : o < d, \forall i: i = o
\label{ILPTransluc1}
\end{equation}

\begin{equation}
\sum_{j\textbackslash \{i,o\}} Ls_{ij}^{od} = \sum_{j\textbackslash \{i,d\}} Ls_{ji}^{od} \qquad \qquad \qquad \qquad \qquad \qquad \qquad
\forall(o,d) : o < d, \forall i: i \neq o,d
\label{ILPTransluc2}
\end{equation}

\begin{equation}
\sum_{j\textbackslash \{d\}} Ls_{ji}^{od} = D_{odc} \qquad \qquad \qquad \qquad \qquad \qquad \qquad \qquad \qquad
\forall(o,d) : o < d, \forall i: i = d
\label{ILPTransluc3}
\end{equation}

\begin{equation}
\sum_{(o,d):o<d} B(c) Ls_{ij}^{od} \leq  100 \lambda_{od} \qquad \qquad \qquad \qquad \qquad \qquad \qquad \qquad \qquad \qquad \qquad 
\forall (i,j)
\label{ILPTransluc4}
\end{equation}

\begin{equation}
\sum_{j\textbackslash \{o\}} f_{ij}^{od} = \lambda_{od}  \qquad \qquad \qquad \qquad \qquad \qquad \qquad \qquad \qquad
\forall(o,d) : o < d, \forall i: i = o
\label{ILPTransluc6}
\end{equation}

\begin{equation}
\sum_{j\textbackslash \{o\}} f_{ij}^{od} = \sum_{j\textbackslash \{d\}} f_{ji}^{od} \qquad \qquad \qquad \qquad \qquad \qquad \qquad \qquad
\forall(o,d) : o < d, \forall i: i \neq o,d
\label{ILPTransluc7}
\end{equation}

\begin{equation}
\sum_{j\textbackslash \{d\}} f_{ji}^{od} = \lambda_{od}  \qquad \qquad \qquad \qquad \qquad \qquad \qquad \qquad \qquad
\forall(o,d) : o < d, \forall i: i = d
\label{ILPTransluc8}
\end{equation}

\begin{equation}
\sum_{(o,d):o<d} \left( f_{ij}^{od} + f_{ji}^{od}\right) \leq K_{ij} G_{ij}
\qquad \qquad \qquad \qquad \qquad \qquad \qquad
\forall (i,j) : i < j
\label{ILPTransluc9}
\end{equation}

\begin{equation}
L_{ij}^{od} \geq 0;
\qquad \qquad \qquad \qquad \qquad \qquad \qquad \qquad \qquad \qquad
\forall (i,j) , \forall (o,d) : o < d
\label{ILPTransluc5}
\end{equation}

\begin{equation}
f_{ij}^{od} \geq 0
\qquad \qquad \qquad \qquad \qquad \qquad \qquad \qquad \qquad \qquad \qquad \qquad
\forall (i,j) \forall (o,d)
\label{ILPTransluc10}
\end{equation}	

