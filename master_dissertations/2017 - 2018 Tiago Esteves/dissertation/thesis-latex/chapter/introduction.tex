\chapter{Introduction}
\label{introduction}

The amount of traffic, in particular IP traffic, has been increasing very substantially. This increase is due to the growing number of Internet-based applications, the increase in the number of devices connected to the Internet, the expansion of optical fiber to customers' homes, increased bandwidth of mobile access technologies, and increased of video traffic \cite{cisco}.
At the same time, with the increase in traffic, operators are under heavy pressure to reduce the cost per bit transported \cite{alcatel_lucent}. This implies the introduction of new technologies, which on the one hand increase the capacity of transport of the networks and on the other, reduce the costs of operation (OPEX) \cite{opexcapex}.
This process of technological conversion is operating in a macroeconomic scenario in which operators find it difficult to finance which forces them to have strong investment constraints (CAPEX) \cite{opexcapex}.
The transport networks have been predominantly based on circuit switching, either at the level of the optical channels or at the level of the electrical circuits, and the introduction of packet switching undermines this paradigm.


\newpage
%%%%%%%%%%%%%%%%%%%%%%%%%%%%%%%%%%%%%%%%%%%%%%%%%%%%%%%%%%%%%%%%%%%%%%%%
\section{Motivation and objectives}
\label{objectives}
Taking into account all these factors, the need to implement planning tools becomes important both for suppliers and operators and is used in the various stages of the telecommunications business.
These have a very important role and directly affect the competitiveness of operators.
One of the tools used for transport network planning is the integer linear programming models.
These models offer optimal solutions, however, some scalability limitations may arise. They also allow quick and easy changes.
Therefore this model becomes relevant in an environment where requirements may differ substantially between operators \cite{teserui}.\\
Due to the importance of transport network planning and design, this dissertation aims to achieve the following main objectives:

\begin{enumerate}
  \item Define one reference network and three different scenarios for performing tests.
  \item Develop ILP models for opaque, transparent and translucent networks without survivability and using 1+1 protection.
  \item Get analytical solutions for the previous point.
  \item Compare the analytical results and results based on ILP with the results obtained through heuristics.
\end{enumerate}


%%%%%%%%%%%%%%%%%%%%%%%%%%%%%%%%%%%%%%%%%%%%%%%%%%%%%%%%%%%%%%%%%%%%%%%%%%%%%%%%%%%%%%%
\section{Thesis outline}
\label{outline}

This thesis is organized in 7 chapters. Chapter \ref{chap_reference_network} consists of a state-of-art review about optical transport networks. In this chapter is also where the reference network used throughout the dissertation as well as the different traffics used is defined. The Chapter \ref{chap_capex} begins by determining the CAPEX calculation formula for use in the ILP model and for analytical calculations. The first section refers to ILP models and the other to analytical models. In Chapter \ref{chap_ilp} are several sections each for a particular mode of transport and certain survivability. In section \ref{ILP_Opaque_Survivability} we have opaque without survivability, in section \ref{ILP_Opaque_Protection} opaque with 1+1 protection. Sections \ref{ILP_Transp_Survivability} and \ref{ILP_Transp_Protection} relate to the transparent and lastly sections \ref{ILP_Transluc_Survivability} and \ref{ILP_Transluc_Protection} refer to the translucent. In the referred section it is possible to see the model description, the detailed description of the results and the conclusions of these results. The analytical calculation of all the models referred to in Chapter \ref{chap_ilp} can be found in Chapter \ref{chap_analytical}. In Chapter \ref{chap_comparative} the results obtained throughout this dissertation are compared and the chapter is divided into six sections where each corresponds to a certain mode of transport with their respective survivability. The last step is the conclusions \ref{chap_conclusions} and suggestions for future research directions.

%%%%%%%%%%%%%%%%%%%%%%%%%%%%%%%%%%%%%%%%%%%%%%%%%%%%%%%%%%%%%%%%%%%%%%%%%%%%%%%%%%%%%%%%%%%%%%%%%%%%%%%%%%%%%%%%%%%%%%%%%%%%%

% References
\phantomsection
\addcontentsline{toc}{section}{References}
%
\renewcommand{\bibname}{References}
%\bibliographystyle{IEEEtran}
%\bibliography{rmorais}
%
%
% Generated by IEEEtran.bst, version: 1.13 (2008/09/30)
\begin{thebibliography}{10}
\providecommand{\url}[1]{#1}
\csname url@samestyle\endcsname
\providecommand{\newblock}{\relax}
\providecommand{\bibinfo}[2]{#2}
\providecommand{\BIBentrySTDinterwordspacing}{\spaceskip=0pt\relax}
\providecommand{\BIBentryALTinterwordstretchfactor}{4}
\providecommand{\BIBentryALTinterwordspacing}{\spaceskip=\fontdimen2\font plus
\BIBentryALTinterwordstretchfactor\fontdimen3\font minus
  \fontdimen4\font\relax}
\providecommand{\BIBforeignlanguage}[2]{{%
\expandafter\ifx\csname l@#1\endcsname\relax
\typeout{** WARNING: IEEEtran.bst: No hyphenation pattern has been}%
\typeout{** loaded for the language `#1'. Using the pattern for}%
\typeout{** the default language instead.}%
\else
\language=\csname l@#1\endcsname
\fi
#2}}
\providecommand{\BIBdecl}{\relax}
\BIBdecl

\bibitem{cisco}
Cisco, ``Global Mobile Data Traffic Forecast Update 2015-2020,'' in \emph{Cisco Visual Networking Index}, pp. 2,3, 2016.

\bibitem{alcatel_lucent}
\BIBentryALTinterwordspacing
Alcatel-Lucent (2009). ``The new economics of telecom networks - bringing value back to the network,'' Tech. Rep. [Online]. Available:
  \url{http://images.tmcnet.com/online-communities/ngc/pdfs/application-enablement/whitepapers/The-New-Economics-of-Telecom-Networks.pdf}
\BIBentrySTDinterwordspacing

\bibitem{opexcapex}
S.~Verbrugge, D.~Colle, M.~Pickavet, P.~Demeester, S.~Pasqualini, A.~Iselt, A.~Kirst\"{a}dter, R.~H\"{u}lsermann, F.-J. Westphal, and M.~J\"{a}ger, ``Methodology and input availability parameters for calculating OpEx and CapEx costs for realistic network scenarios,'' \emph{Journal of Optical Networking}, vol.~5, no.~6, pp. 509--520, June 2006.

\bibitem{teserui}
R.~M.~D. Morais, ``Planning and Dimensioning of Multilayer Optical Transport Networks.'' PhD thesis, Universidade de Aveiro, 2015.

\end{thebibliography}
