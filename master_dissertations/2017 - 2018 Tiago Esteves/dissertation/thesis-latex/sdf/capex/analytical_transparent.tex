
\subsection{Transparent transport mode}\label{analytical_Transp_Mode}

Once more, we must take into account the particularities of this mode of transport before executing the equation of the variables:
\begin{itemize}
  \item $\xi$ = 1.25
  \item $<k>$ = 0 or $<k>$ = $<kp>$ (depending of survivability)
\end{itemize}

The first particularity exists because we are assuming that the grooming coefficient has value 1.25 and finally in the last particularity we are assuming that the survivability coefficient is zero because it is without survivability or $<kp>$ when it is with 1+1 protection \cite{aulas} where

\begin{equation}
<kp> = \frac{<h'>}{<h>}
\label{coefficient_protec2}
\end{equation}

\vspace{13pt}
Finally looking at the equation \ref{analytical_electricalCost} we can see that we already have practically all the values with the exception of three variables. The tributary ports, $P_{TRIB}$, can be calculated through the ODU's matrices referred to in section \ref{Reference_Network_Traffic}, the average number of ports the electrical switch,$<P_{exc}>$, that can be calculated as

\begin{equation}
<P_{exc}> = <d>
\label{Pexc_transp}
\end{equation}

\noindent
and the average number of ports the optical switch,$<P_{oxc}>$, can be calculated as

\begin{equation}
<P_{oxc}> = <d> [1 + \left(1 + <k>\right) <h>]
\label{Poxc_transp}
\end{equation}

\noindent
where $<d>$ is the average number of demands, $<k>$	is the survivability coefficient and $<h>$ is the average number of hops.\\
The number of ports of the electrical switch, in this case, is equal to the number of add ports since we already know the number of tributary ports. The number of ports of the optical switch, in this case, is equal to the sum of the line ports with the add ports \cite{aulas}.\\
