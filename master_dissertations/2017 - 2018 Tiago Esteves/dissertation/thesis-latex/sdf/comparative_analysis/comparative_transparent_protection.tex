\clearpage

\section{Transparent with 1+1 protection}\label{comparative_Transp_Protection}

Once more, for a better analysis of the results, table \ref{table_comparative_transp_protec} was created, with all the scenarios used where it is possible to see values obtained and their margin of error for the ILP model. The first is the dimensioning using ILPs \ref{ILP_Transp_Protection}, the second through analytical models \ref{analytical_Transp_Protection} and finally using heuristic algorithms following a guide document \cite{tesevasco}. It is possible to see these results in detail in the appendices.\\



\begin{table}[h!]
\centering
\begin{tabular}{| c | c | c | c | c |}
 \hline
 \multicolumn{2}{| c |}{ } & ILP & Analytical & Heuristic \\
 \hline\hline
 \multirow{3}{*}{\textbf{Low Traffic}} & Link Cost & 68 520 000 \euro & 24 000 800 \euro & 68 520 000 \euro \\
  & Node Cost & 3 947 590 \euro & 1 448 333 \euro & 4 007 590 \euro \\
  & CAPEX & \textbf{72 467 590 \euro} & \textbf{25 449 133 \euro} (65\%)& \textbf{72 527 590 \euro} (0,1\%)\\
  \hline
 \hline
 \multirow{3}{*}{\textbf{Medium Traffic}} & Link Cost & 226 520 000 \euro & 226 198 400 \euro & 226 520 000 \euro \\
  & Node Cost & 13 020 900 \euro & 12 863 336 \euro & 15 890 900 \euro \\
  & CAPEX & \textbf{239 540 900 \euro*} & \textbf{239 061 736 \euro} (0,2\%)& \textbf{242 410 900 \euro} (1,2\%)\\
 \hline
 \hline
 \multirow{3}{*}{\textbf{High Traffic}} & Link Cost & 424 520 000 \euro & 450 572 800 \euro & 424 520 000 \euro \\
  & Node Cost & 24 286 800 \euro & 25 546 673 \euro & 29 821 800 \euro \\
  & CAPEX & \textbf{448 806 800 \euro*} & \textbf{476 119 473 \euro} (6\%)& \textbf{454 341 800 \euro} (1,2\%)\\
 \hline
\end{tabular}
\caption{Transparent with 1+1 protection: Table with different value of CAPEX for all scenarios.}
\label{table_comparative_transp_protec}
\end{table}

\vspace{13pt}
Comparing the ILP model with the analytical model for this transport mode with 1 + 1 protection there is a very high margin of error (approximately 65\%) for the low scenario. This error happens again for the same reason as above. In this ILP model the coefficient of grooming varies and in this case this value is once again much higher than the analytic one.
For the other two scenarios, as previously mentioned, due to its complexity the model was only executed during two weeks. After these two weeks is presented the best result found so far, which may be the optimal cost or not.
Still in relation to the analytical mode, for the remaining scenarios, it is possible to conclude that it has a much lower margin of error (below 10\%).
Equating to the heuristic model it is possible to observe that the result obtained by the ILP model is always better (smaller) than the value obtained through the heuristic.
For the medium and high scenarios, although it is not possible to guarantee that the indicated value is optimal, it is possible to affirm that it is quite close since, as previously mentioned, it maintains a margin lower than 10\% (compared to the analytic) and obtained a lower value in relation to heuristic.
