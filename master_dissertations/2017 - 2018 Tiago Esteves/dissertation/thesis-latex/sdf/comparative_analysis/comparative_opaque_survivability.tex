\clearpage

\section{Opaque without survivability}\label{comparative_Opaque_Survivability}

In this section, we will compare the CAPEX values obtained for the three scenarios in the three types of design. The first is the dimensioning using ILPs \ref{ILP_Opaque_Survivability}, the second through analytical models \ref{analytical_Opaque_Survivability} and finally using heuristic algorithms following a guide document \cite{tesevasco}. It is possible to see these results in detail in the appendices.\\
For a better analysis of the results, table \ref{table_comparative_opaque_sur} was created, with all the scenarios used where it is possible to see values obtained and their margin of error for the ILP model.\\


\begin{table}[h!]
\centering
\begin{tabular}{| c | c | c | c | c |}
 \hline
 \multicolumn{2}{| c |}{ } & ILP & Analytical & Heuristic \\
 \hline\hline
 \multirow{3}{*}{\textbf{Low Traffic}} & Link Cost & 9 404 000 \euro & 8 520 000 \euro & 12 020 000 \euro \\
  & Node Cost & 1 862 590 \euro & 1 595 720 \euro & 2 362 590 \euro \\
  & CAPEX & \textbf{11 266 590 \euro} & \textbf{10 115 720 \euro} (10\%)& \textbf{14 382 590 \euro} (28\%)\\
 \hline
 \hline
 \multirow{3}{*}{\textbf{Medium Traffic}} & Link Cost & 75 520 000 \euro & 77 520 000 \euro & 77 020 000 \euro \\
  & Node Cost & 15 085 900 \euro & 15 417 260 \euro & 15 385 900 \euro \\
  & CAPEX & \textbf{90 605 900 \euro} & \textbf{92 937 260 \euro} (3\%)& \textbf{92 405 900 \euro} (2\%)\\
 \hline
 \hline
 \multirow{3}{*}{\textbf{High Traffic}} & Link Cost & 148 520 000 \euro & 154 020 000 \euro & 149 020 000 \euro \\
  & Node Cost & 29 711 800 \euro & 30 774 340 \euro & 29 814 200 \euro \\
  & CAPEX & \textbf{178 231 800 \euro} & \textbf{184 794 340 \euro} (4\%)& \textbf{178 834 200 \euro} (0,3\%)\\
  \hline
\end{tabular}
\caption{Opaque without survivability: Table with different value of CAPEX for all scenarios. }
\label{table_comparative_opaque_sur}
\end{table}

\vspace{13pt}
As expected, in all three scenarios, the result obtained through the ILP model is always better (lower) than the value obtained through heuristics. This happens because with the ILP model we always get the optimal solution while with the heuristics we get an approximation of this solution. We can conclude that the higher the traffic, the lower the difference between the ILP and the heuristics because the traffic increase also increases the variables for the heuristic algorithms. Compared with the analytical value, this comparison can not be done literally because the analytical model works with mean values, so this result may be lower or higher than that obtained in the ILP model. It is possible to conclude that this value always has a margin of error of less than 10\% for low scenario and less than 5\% for the other two scenarios. We can conclude that after obtaining the analytical value if applied the margin of error previously mentioned we know that in this interval is the optimal cost.
