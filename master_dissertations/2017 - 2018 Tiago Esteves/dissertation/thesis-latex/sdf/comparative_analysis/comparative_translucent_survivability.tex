\clearpage

\section{Translucent without survivability}\label{comparative_Transluc_Survivability}

In this section, we will compare the CAPEX values obtained for the three scenarios in the two types of design. The first is the dimensioning using ILPs \ref{ILP_Transluc_Survivability} and the second using heuristic algorithms following a guide document \cite{tesevasco}. It is possible to see these results in detail in the appendices. For this case it was not possible to obtain analytical values for comparison.\\
For a better analysis of the results, table \ref{table_comparative_translucent_sur} was created, with all the scenarios used where it is possible to see values obtained and their margin of error for the ILP model.\\


\begin{table}[h!]
\centering
\begin{tabular}{| c | c | c | c |}
 \hline
 \multicolumn{2}{| c |}{ } & ILP & Heuristic \\
 \hline\hline
 \multirow{3}{*}{\textbf{Low Traffic}} & Link Cost & 6 294 000 \euro & 9 520 000 \euro \\
  & Node Cost & 1 237 590 \euro & 2 072 590 \euro \\
  & CAPEX & \textbf{7 531 590 \euro} & \textbf{11 592 590 \euro} (53,9\%)\\
 \hline
 \hline
 \multirow{3}{*}{\textbf{Medium Traffic}} & Link Cost & 36 482 000 \euro & 40 520 000 \euro \\
  & Node Cost & 6 945 900 \euro & 8 605 900 \euro \\
  & CAPEX & \textbf{43 427 900 \euro} & \textbf{49 125 900 \euro} (13,2\%)\\
 \hline
 \hline
 \multirow{3}{*}{\textbf{High Traffic}} & Link Cost & 72 482 000 \euro & 77 520 000 \euro \\
  & Node Cost & 13 506 800 \euro & 16 401 800 \euro \\
  & CAPEX & \textbf{85 988 800 \euro} & \textbf{93 921 800 \euro} (9,2\%)\\
  \hline
\end{tabular}
\caption{Translucent without survivability: Table with different value of CAPEX for all scenarios. }
\label{table_comparative_translucent_sur}
\end{table}


\vspace{13pt}
As already mentioned it is not possible to make comparisons between ILP and analytical calculations. As expected, the results obtained by the ILP model are always better than the values obtained through the heuristic. Comparing the ILP model with the heuristic model for this particular case, we note that, for the low scenario, there is a larger margin of error, approximately 54\%, than for the medium and high scenarios.
In the case of the medium scenario, the heuristic already approaches the optimal cost where the margin of error is lowest, approximately 13\%, which is a big difference compared to the low scenario.
For the high scenario, the heuristic is closer to the ILP model. The value is larger than the ILP model as expected, but with a low margin of error, approximately 9\%.\\
